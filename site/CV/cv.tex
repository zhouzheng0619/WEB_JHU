%!TEX program = xelatex
% Font Size:
%   10pt, 11pt, 12pt
% Paper Size:
%   a4paper, letterpaper, a5paper, leagalpaper, executivepaper, landscape
% Font Family:
%   roman, sans
\documentclass[12pt, a4paper, roman]{moderncv}

% Style:
%   casual, classic, oldstyle, banking
\moderncvstyle{classic}
% Color:
%   blue, orange, green, red, purple, grey
\moderncvcolor{green}
\nopagenumbers{}
%\definecolor{color0}{rgb}{0, 0, 0}
% \definecolor{color1}{RGB}{245, 90, 7}
% \definecolor{color2}{RGB}{39, 40, 34}

% Font specify
\usepackage[UTF8, scheme = plain, heading = false]{ctex}

% Page layout
\usepackage{geometry}
\geometry{scale = 0.8}
% \setlength{\hintscolumnwidth}{4cm}           % 如果你希望改变日期栏的宽度
\AtBeginDocument{\settowidth{\hintscolumnwidth}{XXXX 年 -- XXXX 年}}

\AtBeginDocument{\hypersetup{pdfstartview = FitH}}

% Packages
\usepackage{metalogo}
\usepackage{amsmath}
\usepackage{amsfonts}

\providecommand{\CTeX}{\relax}
\renewcommand{\CTeX}{\ensuremath{\mathbb{C}}\TeX}
\usepackage{paralist}


% Self-info
\name{周}{正}
\title{简  历}
\address{}{黑龙江省鹤岗市}
\email{zhouzheng619@look.com}
\phone[mobile]{15246844859}
% \phone[fixed]{+2~(345)~678~901}
% \phone[fax]{+3~(456)~789~012}
%\homepage{https://github.com/zhouzheng0619}
\extrainfo{性别女 26岁(1991年)}
% \photo[<height>][<width-of-frame>]{<file-name>}
% \photo[64pt][0.4pt]{picture}
% Motto
% \quote{}

% 显示索引号;仅用于在简历中使用了引言
%\makeatletter
%\renewcommand*{\bibliographyitemlabel}{\@biblabel{\arabic{enumiv}}}
%\makeatother

% 分类索引
%\usepackage{multibib}
%\newcites{book,misc}{{Books},{Others}}

\begin{document}
\maketitle
\section{求职意向}
\cventry{}{Web前端开发}{全职/实习}{期待薪资:8k—1.2k/月}{}{}
\section{教育背景}
\cventry{2016.9 -- 2018.1}{理学硕士}{Swansea University}{}{\textit{计算机科学与技术}}{在校期间系统地学习计算机科学与技术知识,熟悉Java和算法理论知识,有数据库基础,了解HTML/CSS和JavaScript}
\cventry{2010.9 -- 2014.6}{文学学士}{中国海洋大学}{}{\textit{德语语言文学}}{熟练掌握德语及英语两种语言知识和技能。具有扎实的德语语言学、德语文学、德汉翻译相关学科的基础知识、复合专业知识。}

\section{毕业论文}
\cvitem{题目}{\emph{Compare Layouts of Hyperlinked Reference: Contextual or Linear?}}
% \cvitem{导师}{导师}
\cvitem{说明}{\small 人机交互方向。通过Java实现PDF文档中超链接的两种布局(单一和比照)并研究这两种模式对电子阅读效率的影响}

\cvitem{题目}{\emph{Der Vergleich zwischen Volksmärchen und Kunstmärchen--Am Beispiel von „Der Geburtstag der Infantin“  und „Schneewittchen“}}
% \cvitem{导师}{导师}
\cvitem{说明}{\small 民间童话和文人童话在情节处理方面的不同:以《西班牙公主的生日》和《白雪公主》为例}


\section{外语技能}
% \section{语言技能}
% \cvitemwithcomment{中文}{母语}{}
\cvitemwithcomment{英语}{熟练}{IELTS:6.0 ;大学英语六级:427}
\cvitemwithcomment{德语}{熟练}{德语专业四级}

\section{工作经验}
\cventry{2015.11——2016.04}{德语笔译}{兼职}{}{}{}
\cventry{2015.03——2015.09}{总编室助理}{后浪(北京)出版咨询有限公司}{}{}{%
\begin{compactitem}
  \item 与出版社合作;
  \item 公司内部编辑事务;
  \item 公司其他部门协调沟通。
  \item 暂代总编辑职务一个月
\end{compactitem}
}
\cventry{2015.02——2015.03}{留学中介德语文书}{兼职}{}{}{}
\cventry{2013.07——2013.09}{鹤岗体育健身中心}{兼职}{}{}{}


% \section{个人兴趣}
% \cvitem{爱好 1}{\small 说明}
% \cvitem{爱好 2}{\small 说明}
% \cvitem{爱好 3}{\small 说明}

% \section{其他 1}
% \cvlistitem{项目 1}
% \cvlistitem{项目 2}
% \cvlistitem{项目 3}

% \renewcommand{\listitemsymbol}{-}             % 改变列表符号

% \section{其他 2}
% \cvlistdoubleitem{项目 1}{项目 4}
% \cvlistdoubleitem{项目 2}{项目 5\cite{book1}}
% \cvlistdoubleitem{项目 3}{}

% % 来自BibTeX文件但不使用multibib包的出版物
% %\renewcommand*{\bibliographyitemlabel}{\@biblabel{\arabic{enumiv}}}% BibTeX的数字标签
% \nocite{*}
% \bibliographystyle{plain}
% \bibliography{publications}                    % 'publications' 是BibTeX文件的文件名

% 来自BibTeX文件并使用multibib包的出版物
%\section{出版物}
%\nocitebook{book1,book2}
%\bibliographystylebook{plain}
%\bibliographybook{publications}               % 'publications' 是BibTeX文件的文件名
%\nocitemisc{misc1,misc2,misc3}
%\bibliographystylemisc{plain}
%\bibliographymisc{publications}               % 'publications' 是BibTeX文件的文件名

\end{document}